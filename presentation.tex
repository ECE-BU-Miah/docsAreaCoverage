<<<<<<< HEAD
% Copyright 2004 by Till Tantau <tantau@users.sourceforge.net>.
%
% In principle, this file can be redistributed and/or modified under
% the terms of the GNU Public License, version 2.
%
% However, this file is supposed to be a template to be modified
% for your own needs. For this reason, if you use this file as a
% template and not specifically distribute it as part of a another
% package/program, I grant the extra permission to freely copy and
% modify this file as you see fit and even to delete this copyright
% notice. 

\documentclass{beamer}

% There are many different themes available for Beamer. A comprehensive
% list with examples is given here:
% http://deic.uab.es/~iblanes/beamer_gallery/index_by_theme.html
% You can uncomment the themes below if you would like to use a different
% one:
%\usetheme{AnnArbor}
%\usetheme{Antibes}
%\usetheme{Bergen}
%\usetheme{Berkeley}
%\usetheme{Berlin}
%\usetheme{Boadilla}
%\usetheme{boxes}
%\usetheme{CambridgeUS}
%\usetheme{Copenhagen}
%\usetheme{Darmstadt}
%\usetheme{default}
%\usetheme{Frankfurt}
%\usetheme{Goettingen}
%\usetheme{Hannover}
%\usetheme{Ilmenau}
%\usetheme{JuanLesPins}
%\usetheme{Luebeck}
\usetheme{Madrid}
%\usetheme{Malmoe}
%\usetheme{Marburg}
%\usetheme{Montpellier}
%\usetheme{PaloAlto}
%\usetheme{Pittsburgh}
%\usetheme{Rochester}
%\usetheme{Singapore}
%\usetheme{Szeged}
%\usetheme{Warsaw}


% Customize Warsaw color 
\setbeamercolor*{palette primary}{use=structure,fg=white,bg=red!50!black}
\setbeamercolor*{palette secondary}{use=structure,fg=white,bg=red!60!black}
\setbeamercolor*{palette tertiary}{use=structure,fg=white,bg=red!70!black}

% Customize Warsaw block title and background colors
\setbeamercolor{block title}{bg=red!50!black,fg=white}

\title[Intro to ROS]{Introduction to Robot Operating System (ROS)}


% % A subtitle is optional and this may be deleted
\subtitle{Application to mobile robots}

\author[A.Elhussein]{Amr~Elhussein  \\\and
Advisor: Dr. Suruz Miah}
% - Give the names in the same order as the appear in the paper.
% - Use the \inst{?} command only if the authors have different
%   affiliation.

\institute[Bradley University] % (optional, but mostly needed)
{
  Department of Electrical and Computer Engineering\\
  Bradley University\\
  1501 W. Bradley Avenue\\
  Peoria, IL, 61625, USA
}
% - Use the \inst command only if there are several affiliations.
% - Keep it simple, no one is interested in your street address.

\date[May~31,~2019]{Friday, May~31,~2019}
% - Either use conference name or its abbreviation.
% - Not really informative to the audience, more for people (including
%   yourself) who are reading the slides online

\logo{\hfill\href{http://www.bradley.edu}{\includegraphics[width=0.75cm]{figs/logoBU1-Print}}}  % place logo in every page 


\subject{Mobile Robot Localization}
% This is only inserted into the PDF information catalog. Can be left
% out. 

% If you have a file called "university-logo-filename.xxx", where xxx
% is a graphic format that can be processed by latex or pdflatex,
% resp., then you can add a logo as follows:

% \pgfdeclareimage[height=0.5cm]{university-logo}{university-logo-filename}
% \logo{\pgfuseimage{university-logo}}

% Delete this, if you do not want the table of contents to pop up at
% the beginning of each subsection:
%\AtBeginSubsection[]
%{
 % \begin{frame}<beamer>{Outline}
  %  \tableofcontents[currentsection,currentsubsection]
  %\end{frame}
%}

% Let's get started
\begin{document}

\begin{frame}
  \titlepage
\end{frame}

\begin{frame}{Outline}
  \tableofcontents
  % You might wish to add the option [pausesections]
\end{frame}

% Section and subsections will appear in the presentation overview
% and table of contents.
\section{Introduction}
\subsection{Historical Background}
\begin{frame}{Intrduction}{History and Legacy}
  
\begin{itemize}
  \item
   Started in 2007 by researches from Stanford AI Robot (Stair) and the Personal Robots (PR) Program and was sponsored by Willow Garage a visionary robotics incubator.
  \item
Used Worlwide in Research and Industry.
  \item
    Currently supported by the Open Source Robotics Foundation.
  \end{itemize}
  \begin{figure}
  \includegraphics[scale=0.2]{figs/img/stair_small}
  \caption{Stair}
  \end{figure}
  
\end{frame}
\subsection{Robot Programming Before ROS}
\begin{frame}{Intrduction}{Robot Programming Before ROS}
\begin{itemize}
  \item
No common platform for develeoping robotics
  \item
Build every thing from scratch 
  \item
  Algorithm implementation 
  \end{itemize}
  
\end{frame}
\subsection{ROS is ..}
\begin{frame}{Intrduction}{ROS is ..}
\begin{block}{}
 A flexible framework for writing robot software. It is a collection of tools, libraries, and conventions that aim to simplify the task of creating complex and robust robot behavior across a wide variety of robotic platforms.
\end{block}

\end{frame}
%----------------------------------
\subsection{ROS Equation}
\begin{frame}{Intrduction}{Ros Equation}
\begin{figure}
\centering
\includegraphics[scale=1.4]{figs/img/ros_equation}
\end{figure}
\end{frame}
%----------------------------------
\subsection{Applications}
\begin{frame}{Intrduction}{Applications}
\begin{figure}
\includegraphics[scale=0.05]{figs/img/selfdrivcar}
\includegraphics[scale=0.4]{figs/img/warehouse}
\includegraphics[scale=0.083]{figs/img/industrial}
\end{figure}

\end{frame}
%----------------------------------  



\section{ROS Concepts}

\subsection{Filesystem}

\begin{frame}{ROS Concepts}{Filesystem}
\begin{figure}
\centering
\includegraphics[scale=0.4]{figs/img/filesystem}
\end{figure}
  
\end{frame}

%----------------------------------

\subsection{Computation Graph}

\begin{frame}{ROS Concepts}{Computation Graph}
\begin{figure}
\centering
\includegraphics[scale=2]{figs/img/computaiongraph}
\end{figure}
  
\end{frame}
%----------------------------------
\begin{frame}{ROS Concepts}{Computation Graph: Master}
\begin{figure}
\centering
\includegraphics[scale=0.35]{figs/img/master}
\end{figure}

\end{frame}
%----------------------------------
\begin{frame}{ROS Concepts}{Computation Graph: Master}
\begin{figure}
\centering
\includegraphics[scale=0.3]{figs/img/nodeseg}
\end{figure}
\end{frame}
%----------------------------------


\subsection{Community level}

\begin{frame}{ROS Concepts}{Community level}
\begin{figure}
\centering
\includegraphics[scale=0.5]{figs/img/community}
\end{figure}
  
\end{frame}

%----------------------------------

\section{ROS installation}
	
\begin{frame}{Installation}

\begin{itemize}
\item Debian-based distributions such as Ubuntu.
\item Many robots.
\item Current supported distributions
\begin{itemize}
\item ROS Kinetic Kame, Released May, 2016.
\item ROS Melodic Morenia, Released May, 2018
%put pics%
\end{itemize}
\end{itemize}

\end{frame}
%----------------------------------
\begin{frame}{Installation}

After choosing the distribution follow the instruction on ROS Wiki which start by:
\begin{itemize}
\item Configure your Ubuntu repositories.
\item Setup your sources.list.
\item Set keys.
\item Install with "sudo apt-get install ros-kinetic-desktop-full".
\end{itemize}

\end{frame}

%----------------------------------

\section{Future of ROS}


\begin{frame}{Future of ROS}

\begin{itemize}
\item Security
\item Critical Missions
\item Distributed Processing
\end{itemize}

\begin{figure}
\includegraphics[scale=0.2]{figs/img/ros2}
\end{figure}  

\end{frame}

%----------------------------------



% You can reveal the parts of a slide one at a time
% with the \pause command:
%\begin{frame}{Second Slide Title}
%  \begin{itemize}
%  \item {
%    First item.
%    \pause % The slide will pause after showing the first item
%  }
  %\item {   
  %  Second item.
 % }
  % You can also specify when the content should appear
  % by using <n->:
 % \item<3-> {
 %   Third item.
 % }
%  \item<4-> {
%    Fourth item.
 % }
  % or you can use the \uncover command to reveal general
  % content (not just \items):
%  \item<5-> {
%    Fifth item. \uncover<6->{Extra text in the fifth item.}
%  }
%  \end{itemize}
%\end{frame}

%\section{Second Main Section}

%\subsection{Another Subsection}

%\begin{frame}{Blocks}
%\begin{block}{Block Title}
%You can also highlight sections of your %presentation in a block, with it's own %title
%\end{block}
%\begin{theorem}
%There are separate environments for %theorems, examples, definitions and proofs.
%\end{theorem}
%\begin{example}
%Here is an example of an example block.
%\end{example}
%\end{frame}

% Placing a * after \section means it will not show in the
% outline or table of contents.
\section*{Questions}
\begin{frame}{}
  \centering \Huge
  \emph{Thanks !}
\end{frame}


\end{document}



%%% Local Variables:
%%% mode: latex
%%% TeX-master: t
%%% End:
\grid
=======
% Copyright 2004 by Till Tantau <tantau@users.sourceforge.net>.
%
% In principle, this file can be redistributed and/or modified under
% the terms of the GNU Public License, version 2.
%
% However, this file is supposed to be a template to be modified
% for your own needs. For this reason, if you use this file as a
% template and not specifically distribute it as part of a another
% package/program, I grant the extra permission to freely copy and
% modify this file as you see fit and even to delete this copyright
% notice. 

\documentclass{beamer}

% There are many different themes available for Beamer. A comprehensive
% list with examples is given here:
% http://deic.uab.es/~iblanes/beamer_gallery/index_by_theme.html
% You can uncomment the themes below if you would like to use a different
% one:
%\usetheme{AnnArbor}
%\usetheme{Antibes}
%\usetheme{Bergen}
%\usetheme{Berkeley}
%\usetheme{Berlin}
%\usetheme{Boadilla}
%\usetheme{boxes}
%\usetheme{CambridgeUS}
%\usetheme{Copenhagen}
%\usetheme{Darmstadt}
%\usetheme{default}
%\usetheme{Frankfurt}
%\usetheme{Goettingen}
%\usetheme{Hannover}
%\usetheme{Ilmenau}
%\usetheme{JuanLesPins}
%\usetheme{Luebeck}
\usetheme{Madrid}
%\usetheme{Malmoe}
%\usetheme{Marburg}
%\usetheme{Montpellier}
%\usetheme{PaloAlto}
%\usetheme{Pittsburgh}
%\usetheme{Rochester}
%\usetheme{Singapore}
%\usetheme{Szeged}
%\usetheme{Warsaw}


% Customize Warsaw color 
\setbeamercolor*{palette primary}{use=structure,fg=white,bg=red!50!black}
\setbeamercolor*{palette secondary}{use=structure,fg=white,bg=red!60!black}
\setbeamercolor*{palette tertiary}{use=structure,fg=white,bg=red!70!black}

% Customize Warsaw block title and background colors
\setbeamercolor{block title}{bg=red!50!black,fg=white}

\title[Intro to ROS]{Introduction to Robot Operating System (ROS)}

% % A subtitle is optional and this may be deleted
% \subtitle{Product Proposal}

\author[A.Elhussein]{Amr~Elhussein  \\\and
Advisor: Dr. Suruz Miah}
% - Give the names in the same order as the appear in the paper.
% - Use the \inst{?} command only if the authors have different
%   affiliation.

\institute[Bradley University] % (optional, but mostly needed)
{
  Department of Electrical and Computer Engineering\\
  Bradley University\\
  1501 W. Bradley Avenue\\
  Peoria, IL, 61625, USA
}
% - Use the \inst command only if there are several affiliations.
% - Keep it simple, no one is interested in your street address.

\date[May~31,~2019]{Friday, May~31,~2019}
% - Either use conference name or its abbreviation.
% - Not really informative to the audience, more for people (including
%   yourself) who are reading the slides online


\setbeamertemplate{bibliography item}{\insertbiblabel}  % insert bibliography numbers instead of symbol
\setbeamertemplate{caption}[numbered] % adds the figure or table number to the caption.

\logo{\hfill\href{http://www.bradley.edu}{\includegraphics[width=0.75cm]{figs/logoBU1-Print}}}  % place logo in every page 


\subject{Mobile Robot Localization}
% This is only inserted into the PDF information catalog. Can be left
% out. 

% If you have a file called "university-logo-filename.xxx", where xxx
% is a graphic format that can be processed by latex or pdflatex,
% resp., then you can add a logo as follows:

% \pgfdeclareimage[height=0.5cm]{university-logo}{university-logo-filename}
% \logo{\pgfuseimage{university-logo}}

% Delete this, if you do not want the table of contents to pop up at
% the beginning of each subsection:
%\AtBeginSubsection[]
%{
 % \begin{frame}<beamer>{Outline}
  %  \tableofcontents[currentsection,currentsubsection]
  %\end{frame}
%}

% Let's get started
\begin{document}

\begin{frame}
  \titlepage
\end{frame}

\begin{frame}{Outline}
  \tableofcontents
  % You might wish to add the option [pausesections]
\end{frame}

% Section and subsections will appear in the presentation overview
% and table of contents.
\section{Introduction}
\subsection{Historical Background}
\begin{frame}{Intrduction}{History and Legacy}
  
\begin{itemize}
  \item
   Started in 2007 by researches from Stanford AI Robot (Stair) and the Personal Robots (PR) Program and was sponsored by Willow Garage a visionary robotics incubator.
  \item
Used Worlwide in Research and Industry.
  \item
    Currently supported by the Open Source Robotics Foundation.
  \end{itemize}
  \begin{figure}
  
  % \includegraphics[scale=0.2]{figs/stair_small}
  \caption{Stair}
  \end{figure}
  
\end{frame}
\subsection{Robot Programming Before ROS}
\begin{frame}{Intrduction}{Robot Programming Before ROS}
\begin{itemize}
  \item
There was no common platform for develeoping robotics and it took ages !
  \item
The developed software for a robot couldn't be used in any other robot.
  \item
  Developers had to implement algorithms on their own.
  \end{itemize}
  
\end{frame}
\subsection{ROS is ..}
\begin{frame}{Intrduction}{ROS is ..}
A flexible framework for writing robot software. It is a collection of tools, libraries, and conventions that aim to simplify the task of creating complex and robust robot behavior across a wide variety of robotic platforms.
  
\end{frame}
\subsection{ROS Equation}
\begin{frame}{Intrduction}{Ros Equation}
\begin{figure}
\centering

% \includegraphics[scale=1.4]{figs/ros_equation}
\end{figure}

  
\end{frame}
%----------------------------------

\section{ROS Concepts}

\subsection{Filesystem}

\begin{frame}{ROS Concepts}{Filesystem}
\begin{figure}
\centering
% \includegraphics[scale=0.4]{figs/filesystem}
\end{figure}
  
\end{frame}

%----------------------------------

\subsection{Computation Graph}

\begin{frame}{ROS Concepts}{Computation Graph}
\begin{figure}
\centering
% \includegraphics[scale=2]{figs/computaiongraph}
\end{figure}
  
\end{frame}
%----------------------------------
\begin{frame}{ROS Concepts}{Computation Graph: Master}
\begin{figure}
\centering
% \includegraphics[scale=0.35]{figs/master}
\end{figure}

\end{frame}
%----------------------------------
\begin{frame}{ROS Concepts}{Computation Graph: Master}
\begin{figure}
\centering
% \includegraphics[scale=0.3]{figs/nodeseg}
\end{figure}
\end{frame}
%----------------------------------


\subsection{Community level}

\begin{frame}{ROS Concepts}{Community level}
\begin{figure}
\centering
% \includegraphics[scale=0.5]{figs/community}
\end{figure}
  
\end{frame}

%----------------------------------

\section{ROS installation}

% put a slide with three dimensional system architecture drawing using ipe
% another slide with explanation

% put a slide with system block diagram

%\subsection{Block Diagram}

\begin{frame}{Installation}

\begin{itemize}
\item Supported for debian-based distributions such as Ubuntu.
\item Supported by many robots.
\item Current supported distributions
\begin{itemize}
\item ROS Kinetic Kame, Released May, 2016.
\item ROS Melodic Morenia, Released May, 2018
%put pics%
\end{itemize}
\end{itemize}

\end{frame}
%----------------------------------
\begin{frame}{Installation}

After choosing the distribution follow the instruction on ROS wiki which start by:
\begin{itemize}
\item Configure your Ubuntu repositories.
\item Setup your sources.list.
\item Set keys
\item Install with "sudo apt-get install ros-kinetic-desktop-full"
\end{itemize}

\end{frame}

%----------------------------------

\section{Future of ROS}


\begin{frame}{Future of ROS}
  
\end{frame}

%----------------------------------



% You can reveal the parts of a slide one at a time
% with the \pause command:
%\begin{frame}{Second Slide Title}
%  \begin{itemize}
%  \item {
%    First item.
%    \pause % The slide will pause after showing the first item
%  }
  %\item {   
  %  Second item.
 % }
  % You can also specify when the content should appear
  % by using <n->:
 % \item<3-> {
 %   Third item.
 % }
%  \item<4-> {
%    Fourth item.
 % }
  % or you can use the \uncover command to reveal general
  % content (not just \items):
%  \item<5-> {
%    Fifth item. \uncover<6->{Extra text in the fifth item.}
%  }
%  \end{itemize}
%\end{frame}

%\section{Second Main Section}

%\subsection{Another Subsection}

%\begin{frame}{Blocks}
%\begin{block}{Block Title}
%You can also highlight sections of your %presentation in a block, with it's own %title
%\end{block}
%\begin{theorem}
%There are separate environments for %theorems, examples, definitions and proofs.
%\end{theorem}
%\begin{example}
%Here is an example of an example block.
%\end{example}
%\end{frame}

% Placing a * after \section means it will not show in the
% outline or table of contents.
\section*{Summary}

\begin{frame}{Summary}
  \begin{itemize}
  \item
    The \alert{first main message} of your talk in one or two lines.
  \item
    The \alert{second main message} of your talk in one or two lines.
  \item
    Perhaps a \alert{third message}, but not more than that.
  \end{itemize}
  
  \begin{itemize}
  \item
    Outlook
    \begin{itemize}
    \item
      Something you haven't solved.
    \item
      Something else you haven't solved.
    \end{itemize}
  \end{itemize}
\end{frame}



% All of the following is optional and typically not needed. 
\appendix
\section<presentation>*{\appendixname}
\subsection<presentation>*{For Further Reading}

\begin{frame}[allowframebreaks]
  \frametitle<presentation>{For Further Reading}
    
  \begin{thebibliography}{10}
    
  \beamertemplatebookbibitems
  % Start with overview books.

  \bibitem{Author1990}
    A.~Author.
    \newblock {\em Handbook of Everything}.
    \newblock Some Press, 1990.
 
    
  \beamertemplatearticlebibitems
  % Followed by interesting articles. Keep the list short. 

  \bibitem{Someone2000}
    S.~Someone.
    \newblock On this and that.
    \newblock {\em Journal of This and That}, 2(1):50--100,
    2000.
  \end{thebibliography}
\end{frame}

\end{document}



%%% Local Variables:
%%% mode: latex
%%% TeX-master: t
%%% End:
\grid
>>>>>>> c5f9b73c734c9459960d2f7c9a0502b8deaa9ddd
