%%%%%%%%%%%%%%%%%%%%%%%%%%%%%%%%%%%%%%%%
% Compact Laboratory Book
% LaTeX Template
% Version 1.0 (4/6/12)
%
% This template has been downloaded from:
% http://www.LaTeXTemplates.com
%
% Original author:
% Joan Queralt Gil (http://phobos.xtec.cat/jqueralt) using the labbook class by
% Frank Kuster (http://www.ctan.org/tex-archive/macros/latex/contrib/labbook/)
%
% License:
% CC BY-NC-SA 3.0 (http://creativecommons.org/licenses/by-nc-sa/3.0/)
%
% Important note:
% This template requires the labbook.cls file to be in the same directory as the
% .tex file. The labbook.cls file provides the necessary structure to create the
% lab book.
%
% The \lipsum[#] commands throughout this template generate dummy text
% to fill the template out. These commands should all be removed when 
% writing lab book content.
%
% HOW TO USE THIS TEMPLATE 
% Each day in the lab consists of three main things:
%
% 1. LABDAY: The first thing to put is the \labday{} command with a date in 
% curly brackets, this will make a new section showing that you are working
% on a new day.
%
% 2. EXPERIMENT/SUBEXPERIMENT: Next you need to specify what 
% experiment(s) and subexperiment(s) you are working on with a 
% \experiment{} and \subexperiment{} commands with the experiment 
% shorthand in the curly brackets. The experiment shorthand is defined in the 
% 'DEFINITION OF EXPERIMENTS' section below, this means you can 
% say \experiment{pcr} and the actual text written to the PDF will be what 
% you set the 'pcr' experiment to be. If the experiment is a one off, you can 
% just write it in the bracket without creating a shorthand. Note: if you don't 
% want to have an experiment, just leave this out and it won't be printed.
%
% 3. CONTENT: Following the experiment is the content, i.e. what progress 
% you made on the experiment that day.
%
%%%%%%%%%%%%%%%%%%%%%%%%%%%%%%%%%%%%%%%%%

%----------------------------------------------------------------------------------------
%	PACKAGES AND OTHER DOCUMENT CONFIGURATIONS
%----------------------------------------------------------------------------------------                               
%\UseRawInputEncoding
\documentclass[fontsize=11pt, % Document font size
                             paper=letter, % Document paper type
                             twoside, % Shifts odd pages to the left for easier reading when printed, can be changed to oneside
                             captions=tableheading,
                             index=totoc,
                             hyperref]{labbook}

%\documentclass[idxtotoc,hyperref,openany]{labbook} % 'openany' here removes the
   
%\usepackage[hyperref]
\usepackage[bottom=10em]{geometry} % Reduces the whitespace at the bottom of the page so more text can fit

\usepackage[english]{babel} % English language
\usepackage{lipsum} % Used for inserting dummy 'Lorem ipsum' text into the template

\usepackage[utf8]{inputenc} % Uses the utf8 input encoding
\usepackage[T1]{fontenc} % Use 8-bit encoding that has 256 glyphs

\usepackage[osf]{mathpazo} % Palatino as the main font
\linespread{1.05}\selectfont % Palatino needs some extra spacing, here 5% extra
\usepackage[scaled=.88]{beramono} % Bera-Monospace
\usepackage[scaled=.86]{berasans} % Bera Sans-Serif

\usepackage{booktabs,array} % Packages for tables

\usepackage{amsmath} % For typesetting math
\usepackage{graphicx} % Required for including images
\usepackage{etoolbox}
\usepackage[norule]{footmisc} % Removes the horizontal rule from footnotes
\usepackage{lastpage} % Counts the number of pages of the document

\usepackage[dvipsnames]{xcolor}  % Allows the definition of hex colors

\definecolor{titleblue}{rgb}{0.16,0.24,0.64} % Custom color for the title on the title page
\definecolor{linkcolor}{rgb}{0,0,0.42} % Custom color for links - dark blue at the moment

\addtokomafont{title}{\Huge\color{titleblue}} % Titles in custom blue color
\addtokomafont{chapter}{\color{OliveGreen}} % Lab dates in olive green
\addtokomafont{section}{\color{Sepia}} % Sections in sepia
\addtokomafont{pagehead}{\normalfont\sffamily\color{gray}} % Header text in gray and sans serif
\addtokomafont{caption}{\footnotesize\itshape} % Small italic font size for captions
\addtokomafont{captionlabel}{\upshape\bfseries} % Bold for caption labels
\addtokomafont{descriptionlabel}{\rmfamily}
\setcapwidth[r]{10cm} % Right align caption text
\setkomafont{footnote}{\sffamily} % Footnotes in sans serif

\deffootnote[4cm]{4cm}{1em}{\textsuperscript{\thefootnotemark}} % Indent footnotes to line up with text

\DeclareFixedFont{\textcap}{T1}{phv}{bx}{n}{1.5cm} % Font for main title: Helvetica 1.5 cm
\DeclareFixedFont{\textaut}{T1}{phv}{bx}{n}{0.8cm} % Font for author name: Helvetica 0.8 cm

\usepackage[nouppercase,headsepline]{scrpage2} % Provides headers and footers configuration
\pagestyle{scrheadings} % Print the headers and footers on all pages
\clearscrheadfoot % Clean old definitions if they exist

\automark[chapter]{chapter}
\ohead{\headmark} % Prints outer header

\setlength{\headheight}{25pt} % Makes the header take up a bit of extra space for aesthetics
\setheadsepline{.4pt} % Creates a thin rule under the header
\addtokomafont{headsepline}{\color{lightgray}} % Colors the rule under the header light gray

\ofoot[\normalfont\normalcolor{\thepage\ |\  \pageref{LastPage}}]{\normalfont\normalcolor{\thepage\ |\  \pageref{LastPage}}} % Creates an outer footer of: "current page | total pages"

% These lines make it so each new lab day directly follows the previous one i.e. does not start on a new page - comment them out to separate lab days on new pages
\makeatletter
\patchcmd{\addchap}{\if@openright\cleardoublepage\else\clearpage\fi}{\par}{}{}
\makeatother
\renewcommand*{\chapterpagestyle}{scrheadings}

% These lines make it so every figure and equation in the document is numbered consecutively rather than restarting at 1 for each lab day - comment them out to remove this behavior
\usepackage{chngcntr}
\counterwithout{figure}{labday}
\counterwithout{equation}{labday}

% Hyperlink configuration
\usepackage[
    pdfauthor={}, % Your name for the author field in the PDF
    pdftitle={Laboratory Journal}, % PDF title
    pdfsubject={}, % PDF subject
    bookmarksopen=true,
    linktocpage=true,
    urlcolor=linkcolor, % Color of URLs
    citecolor=linkcolor, % Color of citations
    linkcolor=linkcolor, % Color of links to other pages/figures
    backref=page,
    pdfpagelabels=true,
    plainpages=false,
    colorlinks=true, % Turn off all coloring by changing this to false
    bookmarks=true,
    pdfview=FitB]{hyperref}

\usepackage[stretch=10]{microtype} % Slightly tweak font spacing for aesthetics

%\setlength\parindent{0pt} % Uncomment to remove all indentation from paragraphs

%----------------------------------------------------------------------------------------
%	DEFINITION OF EXPERIMENTS
%----------------------------------------------------------------------------------------

% Template: \newexperiment{<abbrev>}[<short form>]{<long form>}
% <abbrev> is the reference to use later in the .tex file in \experiment{}, the <short form> is only used in the table of contents and running title - it is optional, <long form> is what is printed in the lab book itself

\newexperiment{example}[Example experiment]{This is an example experiment}
\newexperiment{example2}[Example experiment 2]{This is another example experiment}
\newexperiment{example3}[Example experiment 3]{This is yet another example experiment}

\newsubexperiment{subexp_example}[Example sub-experiment]{This is an example sub-experiment}
\newsubexperiment{subexp_example2}[Example sub-experiment 2]{This is another example sub-experiment}
\newsubexperiment{subexp_example3}[Example sub-experiment 3]{This is yet another example sub-experiment}

%----------------------------------------------------------------------------------------
\newcommand{\HRule}{\rule{\linewidth}{0.5mm}} % Command to make the lines in the title page

\setlength\parindent{0pt} % Removes all indentation from paragraphs

\begin{document}

%----------------------------------------------------------------------------------------
%	TITLE PAGE
%----------------------------------------------------------------------------------------
%\frontmatter % Use Roman numerals for page numbers

%\begin{center}

%

\title{
\begin{center}
\href{http://www.bradley.edu}{\includegraphics[height=0.5in]{figs/logoBU1-Print}}
\vskip10pt
\HRule \\[0.4cm]
{\Huge \bfseries Project Notebook \\[0.5cm] \Large Area Coverage Optimization}\\[0.4cm] % Degree
\HRule \\[1.5cm]
\end{center}
}
\author{\Huge Amr Elhussein \\ \\ \LARGE aelhussein@mail.bradley.edu \\[2cm]} % Your name and email address
\date{Beginning May 2019} % Beginning date
\maketitle

%\maketitle % Title page

\printindex
\tableofcontents % Table of contents
\newpage % Start lab look on a new page

\begin{addmargin}[0cm]{0cm} % Makes the text width much shorter for a compact look

\pagestyle{scrheadings} % Begin using headers

 %----------------------------------------------------------------------------------------




%----------------------------------------------------------------------------------------
%	LAB BOOK CONTENTS
%----------------------------------------------------------------------------------------
\labday{Wednesday, May 08, 2019}
%----------------------------------------------------------------------------------------
Dr. Miah gave me an introduction to the project and what's has been done so far. he walked me through the senior project presentation and documentation.  
%\bigbreak\noindent

%\bigbreak\noindent
 
%\bigbreak\noindent

%\bigbreak\noindent

%\bigbreak\noindent

%\bigbreak\noindent
%I need to learn the following Github Bash terminal commands:
%\begin{verbatim}
%Git add
%Git commit -m "message"
%Git remote add origin 'url'
%\end{verbatim}
%To get a better undestanding of what is going on I need to go through Github tutorials.

%\item Upload latex file to Drive
\labday{Thursday, May 09, 2019}
We had a short meeting with Dakota and Eric and we discussed the prerequisites for the area coverage project and we agreed to meet again to go over the simulation and the implementation of the project.
\bigbreak\noindent
I have to setup the following on my laptop:
\begin{itemize}
\item Ubuntu
\item ROS Kinetic
\item Putty
\item Winscp
\item VRep
\item Matlab robotics toolbox (on the lab's pc)
\end{itemize}


\labday{Friday, May 10, 2019}
I met Eric and Dakota in the robotics lab to transfer the project work. 
\bigbreak\noindent
They walked me through the basics of the project and a quick hadwaving on the simulation and physical implementation. 
\bigbreak\noindent
Dr. Miah wanted me to run at least on Demo on my laptop. However we figured out in order to do that i will need two machines one with ROS and other with matlab toolbox on it. 
\labday{Wed, May 22, 2019}
I met Dr. Miah Regarding the computer lab access and he sent to following email to Mr. Mattus the lab director:
\bigbreak\noindent
Dear Chris,

Amr Elhussein is an ME graduate student and is currently working on a 
robotics project under my direction. He needs access to a computer in 
the robotics lab (JOB 254). I'd appreciate if you could please give him 
access to one of the computers in JOBST 254 as soon as possible. He 
needs to run some simulations immediately.

I know it's moving time but he will be out of the lab before moving starts.

Thanks,
\bigbreak\noindent
The project documentation will be using Latex, and we also set the github account to put all the project material on. 
\bigbreak\noindent
We also set up a weekly presentation session to update the Robotics and mechatronics research group on the progress of the project. 
\bigbreak\noindent
The first two presentations will be :
\begin{itemize}
\item an introduction to ROS. 
\item Matlab Robotics toolbox.
\end{itemize} 
%----------------------------------
\labday{Tue, May 28, 2019}
We had a Skype meeting to discuss the content of my presentation. Dr. Miah reviewed the presentation and he ponted out some remarks which i edited immediatley. 
%------------------------------------------------------------
\labday {Fri, May 7, 2019}
We had our first RAM group meeting, Caleb and Dylan gave a presentation on Mobile Robot Localization using Trilateration, Brian gave a presentation on BEMOSS and its enhanced applications and i gave the presentation on ROS. 
%--------------------------------------------------------
\labday{Tuesday, June 4, 2019}
We had a skype meeting on which we decided to install trial version of matlab on my laptop and use one of the other laptops available in the lab to run v-rep on. 
\bigbreak\noindent
Dr. Miah also wanted me to create a list of available programable quadcopters that we can purchase for our project. 
\bigbreak\noindent
Currently the official ROS website contains the following quadcopters:
\begin{itemize}
\item AscTec Pelican and Hummingbird quadrotors: cuurently unavailable. 

\item Berkeley’s STARMAC: unavailable commercially developed by Berkeley. 
\item Bitcraze Crazyflie: costs around 205.00 
\\
\url {https://www.amazon.com/Seeed-Studio-Crazyflie-V2-1-Quadcopters/dp/B07QLVRXN9/ref=asc_df_B07QLVRXN9/?tag=hyprod-20&linkCode=df0&hvadid=343224652930&hvpos=1o1&hvnetw=g&hvrand=13584647834092143486&hvpone=&hvptwo=&hvqmt=&hvdev=c&hvdvcmdl=&hvlocint=&hvlocphy=1016766&hvtargid=pla-757568426634&psc=1&tag=&ref=&adgrpid=71764766791&hvpone=&hvptwo=&hvadid=343224652930&hvpos=1o1&hvnetw=g&hvrand=13584647834092143486&hvqmt=&hvdev=c&hvdvcmdl=&hvlocint=&hvlocphy=1016766&hvtargid=pla-757568426634
}
\item DJI Matrice 100 Onboard SDK ROS support: mainly for developers but it's very expensive around 3K.
\\
\url {https://www.dji.com/matrice100}
\item Erle-copter: didn't find a commercial one for purchasing. 
\\
\url {http://docs.erlerobotics.com/erle_robots/erle_copter}

\item ETH sFly: commercially unavailable, developeed by ETH zurich Autonomus sytems labs. 
\item Lily CameraQuadrotor : currently unavailable, the project was shut down. 
\item Parrot AR.Drone
Costs around 174.00 
\\
\url{
https://www.amazon.com/Parrot-AR-Drone-Quadricopter-Power/dp/B00D8UP6I0/ref=asc_df_B00D8UP6I0/?tag=hyprod-20&linkCode=df0&hvadid=343187928718&hvpos=1o1&hvnetw=g&hvrand=2299143640651648665&hvpone=&hvptwo=&hvqmt=&hvdev=c&hvdvcmdl=&hvlocint=&hvlocphy=1016766&hvtargid=pla-523888688318&psc=1&tag=&ref=&adgrpid=71716366209&hvpone=&hvptwo=&hvadid=343187928718&hvpos=1o1&hvnetw=g&hvrand=2299143640651648665&hvqmt=&hvdev=c&hvdvcmdl=&hvlocint=&hvlocphy=1016766&hvtargid=pla-523888688318
}
\item Parrot Bebop: costs around 169.99
\\
\url{https://www.amazon.com/dp/B00OOR9060/ref=dp_cr_wdg_tit_nw_mr} 

\item Penn’s AscTec Hummingbird Quadrotors


\end{itemize}

For DJI drones the only ones that are supported by ROS are quite expensive. so we either have to choose from the list or build our own drone. 
%---------------------------------------------------
\labday{Wed, Jun 5, 2019}
I contacted Mathworks.com to get a free trial for matlab robotics system toolbox and i got it installed on ubuntu 16.04 along with ROS. 
\\ The next challenge was to figure out a way to connect matlab robotics toolbox and ROS connected on the same machine as all the tutorials on mathworks website talks about connecting Matlab and ROS on different machines.

%-------------------------------------------`
\labday{Fri, Jun 7, 2019}
We had our Robotics and Mechatronics Reseach Group (RAM) meeting in which Caleb presented a brief introduction on Multi robot loclization using trieleration. Then I presented a brief introduction on Matlab Robotics Toolbox and its intergration with ROS and the challenges that needs to be tackled. 
%-------------------------
\labday{Tue, June 11, 2019}
I solved the problem of integrating Matlab Robotics Toolbox and ROS on the same machine. it appeared that you need to intilize ROS master in the terminal and then in matlab if you typed  :
\\ \begin{verbatim}
rosinit
\end{verbatim} 
 it will connects to that existing ROS master node. 
 %---------------------------------- 
 \labday{Thu, June 13, 2019}
 I followed turtlebot 3 tutorials on mathworks.com to verify the connection between Matlab and ROS. 
 %---------------------------------------------
 \labday{Fri, June 14, 2019}
 We had our weekly meeting via Skype, we discussed my recent progress and we agreed on the next step which is implementing the three demos on my laptop. 
 \\ We set our weekly meetings back to 3:00 pm if no presentation sessions are required. 
 \\ in the Upcoming weeks I'll spend more time in the lab working with the actual robots.
\\Dr. Miah expressed his intent in writing a proposal grant to fund the project in which I'll be part of. 
\\Dr. Miah asked me to check the pricing of the Matlab robotics toolbox for students which i discoverd later was around 100 USD including simulink. 

%-------------------------------------------------
\labday{Fri, June 21, 2019}
Presnted my progress in RAM Group meeting, and i demonstrated the three simulation demos on my laptop.
%------------------------------------------------
\labday{Fri, June 28, 2019}
Started working on the actual robots but didn't know the login and password till the end of lab hours.
\\ I also worked on the cad modeling of the eduMOD robot, waiting for Eric to provide the most recent files to work on. 
%--------------------------------------------------
\labday{Tuesday, July 2, 2019}
Completley modeled the eduMOD robot in solidworks but when it was exported to v-rep nothing is visible. I used SW2URDF plugin.
%------------------------------------------------
\labday {Fri, July 5, 2019}
Presented the 3d model in the RAM group bi-weekly meeting and demonstrated the concept of URDF and my future plans on understanding it in more depth. 
%------------------------------------------
\labday {Fri, July 12,2019}
 Back again to working on the actual robots, after sucessfully interfacing MatlabRobotics Toolbox and the beaglebone i didn't know the names of the packages and executebles to run on the beaglebone as there is no any documentation in that area. 
%----------------------------
 \labday{Tuesday, July 16,2019}
 started working in the lab at 1:30 later on Eric Joined me to discuss over the problems that i was facing, we sucessfully run the linefollowing implementation and i then worked on the leaderfollowing implementation , However regarding the area Coverage algorithm i couldn't find the matlab code to run it. 
 \\ Also a small incident occured where the caster on one of the robots was disattached to the base as the glue that supported it became weak, i told Eric about the problem to see what type of Glue they used.
%----------------------------------
\labday{Fri, July 19,2019}
we had our progress presentation sesssion , Brian, Caleb and I presented our work up to the moment. we also discussed our future plans and deadlines. 
\\ After that i worked on implementing the line follower and leader follower and sucessfully generated the plots and videos. i am still waiting for the area coverage code and i'll also start looking at modifying the simulation code to serve my goal. 
\\Regarding the simulation i also tried what Eric suggested by running . ./vrep.sh when i intially start vrep but for some reason it didn't work. 
\\ I also glued the caster for robot No.2 and waiting for it to dry. 
\end{addmargin}

\labday{FRI, July 26,2019}
I met with Dr. Miah and we discussed the progress of the project, we agreed to give the priority to understanding the actor critic network and its implementation on the area coverage project. 
\\ During the lab hours i worked on modifying the area coverage simulation code to suite implementation purposes, i modified the publishers and subscribes names and data type and i sucesssfully sent data between the three robots and the robotic toolbox, however the robot didn't act as the they should, they kept rotating around them self i am suspecting that is due to some errors in specifying the intial positions of the robots or maybe in updating those values. 
\\while working with robot No.2 i experienced some issues, when first attempting to connect through ssh the beaglbone only had bin and it indicated that this image was from 2017, upon trying multiple times it worked but i probably need to back up the image and then resotre it again. 
%--------------------------------
\labday{Fri, Aug 2, 2019}
I presented my progress in our group meeting and i demonstrated the line follower and line following implementation and i discussed with Dr. Miah and Eric the problem i'm facing with the area coverage algorithm not working sucessfully. 
\\Eric and I head to the lab afterwards to look into the problem more closely and we discovered that the distance between the two wheels is set to the pioneer and also the left and right wheel were flipped in the matlab code, we also edited the gain values and the algorithm worked sucessfully. 
%--------------------------------------
\labday{Fri, Aug 9, 2019}
Dr. Miah ordered a new laptop for the project so i went ahead and edited the codes to work with the new laptop. Unforunatley there was a problem configuring the new laptop with our framework so i called it off that day. 
%-----------------
\labday{Tue, Aug 12, 2019}
I found out that the problem was from the firewall on the new laptop so i asked Dr. Miah to disable the Firewall using adminstrative login and that solved the problem.
\\Before implementing the area coverage algorithm i had a problem with the Beaglebone No.1 which it's SD card suddenly stopped responding which resulted in me not being able to ssh the beaglebone, I backed up the beaglbone No.2 and then restored it on the first one, i changed the topics names and then built the package and then it worked sucessfully. 
%---------------------------------
\labday{Wed, Aug 13, 2019}
I implemented the area coverage algorithm and i took videos for the presentation, I noticed that the plotting function is not updating the positions of the robot and it's stuck in the intial positions so I'll look into it. 
%----------------------------------------------------------------------------------------
%	BIBLIOGRAPHY
%----------------------------------------------------------------------------------------


\bibliographystyle{plain}
\bibliography{bib/seniorProject2017.bib}


% \begin{thebibliography}{9}

% \bibitem{lamport94}
% Leslie Lamport,
% \emph{\LaTeX: A Document Preparation System}.
% Addison Wesley, Massachusetts,
% 2nd Edition,
% 1994.

% \end{thebibliography}

%----------------------------------------------------------------------------------------

\end{document}


%%% Local Variables:
%%% mode: latex
%%% TeX-master: t
%%% End:
\grid
