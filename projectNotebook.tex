%%%%%%%%%%%%%%%%%%%%%%%%%%%%%%%%%%%%%%%%
% Compact Laboratory Book
% LaTeX Template
% Version 1.0 (4/6/12)
%
% This template has been downloaded from:
% http://www.LaTeXTemplates.com
%
% Original author:
% Joan Queralt Gil (http://phobos.xtec.cat/jqueralt) using the labbook class by
% Frank Kuster (http://www.ctan.org/tex-archive/macros/latex/contrib/labbook/)
%
% License:
% CC BY-NC-SA 3.0 (http://creativecommons.org/licenses/by-nc-sa/3.0/)
%
% Important note:
% This template requires the labbook.cls file to be in the same directory as the
% .tex file. The labbook.cls file provides the necessary structure to create the
% lab book.
%
% The \lipsum[#] commands throughout this template generate dummy text
% to fill the template out. These commands should all be removed when 
% writing lab book content.
%
% HOW TO USE THIS TEMPLATE 
% Each day in the lab consists of three main things:
%
% 1. LABDAY: The first thing to put is the \labday{} command with a date in 
% curly brackets, this will make a new section showing that you are working
% on a new day.
%
% 2. EXPERIMENT/SUBEXPERIMENT: Next you need to specify what 
% experiment(s) and subexperiment(s) you are working on with a 
% \experiment{} and \subexperiment{} commands with the experiment 
% shorthand in the curly brackets. The experiment shorthand is defined in the 
% 'DEFINITION OF EXPERIMENTS' section below, this means you can 
% say \experiment{pcr} and the actual text written to the PDF will be what 
% you set the 'pcr' experiment to be. If the experiment is a one off, you can 
% just write it in the bracket without creating a shorthand. Note: if you don't 
% want to have an experiment, just leave this out and it won't be printed.
%
% 3. CONTENT: Following the experiment is the content, i.e. what progress 
% you made on the experiment that day.
%
%%%%%%%%%%%%%%%%%%%%%%%%%%%%%%%%%%%%%%%%%

%----------------------------------------------------------------------------------------
%	PACKAGES AND OTHER DOCUMENT CONFIGURATIONS
%----------------------------------------------------------------------------------------                               
%\UseRawInputEncoding
\documentclass[fontsize=11pt, % Document font size
                             paper=letter, % Document paper type
                             twoside, % Shifts odd pages to the left for easier reading when printed, can be changed to oneside
                             captions=tableheading,
                             index=totoc,
                             hyperref]{labbook}

%\documentclass[idxtotoc,hyperref,openany]{labbook} % 'openany' here removes the
   
\usepackage[bottom=10em]{geometry} % Reduces the whitespace at the bottom of the page so more text can fit

\usepackage[english]{babel} % English language
\usepackage{lipsum} % Used for inserting dummy 'Lorem ipsum' text into the template

\usepackage[utf8]{inputenc} % Uses the utf8 input encoding
\usepackage[T1]{fontenc} % Use 8-bit encoding that has 256 glyphs

\usepackage[osf]{mathpazo} % Palatino as the main font
\linespread{1.05}\selectfont % Palatino needs some extra spacing, here 5% extra
\usepackage[scaled=.88]{beramono} % Bera-Monospace
\usepackage[scaled=.86]{berasans} % Bera Sans-Serif

\usepackage{booktabs,array} % Packages for tables

\usepackage{amsmath} % For typesetting math
\usepackage{graphicx} % Required for including images
\usepackage{etoolbox}
\usepackage[norule]{footmisc} % Removes the horizontal rule from footnotes
\usepackage{lastpage} % Counts the number of pages of the document

\usepackage[dvipsnames]{xcolor}  % Allows the definition of hex colors
\definecolor{titleblue}{rgb}{0.16,0.24,0.64} % Custom color for the title on the title page
\definecolor{linkcolor}{rgb}{0,0,0.42} % Custom color for links - dark blue at the moment

\addtokomafont{title}{\Huge\color{titleblue}} % Titles in custom blue color
\addtokomafont{chapter}{\color{OliveGreen}} % Lab dates in olive green
\addtokomafont{section}{\color{Sepia}} % Sections in sepia
\addtokomafont{pagehead}{\normalfont\sffamily\color{gray}} % Header text in gray and sans serif
\addtokomafont{caption}{\footnotesize\itshape} % Small italic font size for captions
\addtokomafont{captionlabel}{\upshape\bfseries} % Bold for caption labels
\addtokomafont{descriptionlabel}{\rmfamily}
\setcapwidth[r]{10cm} % Right align caption text
\setkomafont{footnote}{\sffamily} % Footnotes in sans serif

\deffootnote[4cm]{4cm}{1em}{\textsuperscript{\thefootnotemark}} % Indent footnotes to line up with text

\DeclareFixedFont{\textcap}{T1}{phv}{bx}{n}{1.5cm} % Font for main title: Helvetica 1.5 cm
\DeclareFixedFont{\textaut}{T1}{phv}{bx}{n}{0.8cm} % Font for author name: Helvetica 0.8 cm

\usepackage[nouppercase,headsepline]{scrpage2} % Provides headers and footers configuration
\pagestyle{scrheadings} % Print the headers and footers on all pages
\clearscrheadfoot % Clean old definitions if they exist

\automark[chapter]{chapter}
\ohead{\headmark} % Prints outer header

\setlength{\headheight}{25pt} % Makes the header take up a bit of extra space for aesthetics
\setheadsepline{.4pt} % Creates a thin rule under the header
\addtokomafont{headsepline}{\color{lightgray}} % Colors the rule under the header light gray

\ofoot[\normalfont\normalcolor{\thepage\ |\  \pageref{LastPage}}]{\normalfont\normalcolor{\thepage\ |\  \pageref{LastPage}}} % Creates an outer footer of: "current page | total pages"

% These lines make it so each new lab day directly follows the previous one i.e. does not start on a new page - comment them out to separate lab days on new pages
\makeatletter
\patchcmd{\addchap}{\if@openright\cleardoublepage\else\clearpage\fi}{\par}{}{}
\makeatother
\renewcommand*{\chapterpagestyle}{scrheadings}

% These lines make it so every figure and equation in the document is numbered consecutively rather than restarting at 1 for each lab day - comment them out to remove this behavior
\usepackage{chngcntr}
\counterwithout{figure}{labday}
\counterwithout{equation}{labday}

% Hyperlink configuration
\usepackage[
    pdfauthor={}, % Your name for the author field in the PDF
    pdftitle={Laboratory Journal}, % PDF title
    pdfsubject={}, % PDF subject
    bookmarksopen=true,
    linktocpage=true,
    urlcolor=linkcolor, % Color of URLs
    citecolor=linkcolor, % Color of citations
    linkcolor=linkcolor, % Color of links to other pages/figures
    backref=page,
    pdfpagelabels=true,
    plainpages=false,
    colorlinks=true, % Turn off all coloring by changing this to false
    bookmarks=true,
    pdfview=FitB]{hyperref}

\usepackage[stretch=10]{microtype} % Slightly tweak font spacing for aesthetics

%\setlength\parindent{0pt} % Uncomment to remove all indentation from paragraphs

%----------------------------------------------------------------------------------------
%	DEFINITION OF EXPERIMENTS
%----------------------------------------------------------------------------------------

% Template: \newexperiment{<abbrev>}[<short form>]{<long form>}
% <abbrev> is the reference to use later in the .tex file in \experiment{}, the <short form> is only used in the table of contents and running title - it is optional, <long form> is what is printed in the lab book itself

\newexperiment{example}[Example experiment]{This is an example experiment}
\newexperiment{example2}[Example experiment 2]{This is another example experiment}
\newexperiment{example3}[Example experiment 3]{This is yet another example experiment}

\newsubexperiment{subexp_example}[Example sub-experiment]{This is an example sub-experiment}
\newsubexperiment{subexp_example2}[Example sub-experiment 2]{This is another example sub-experiment}
\newsubexperiment{subexp_example3}[Example sub-experiment 3]{This is yet another example sub-experiment}

%----------------------------------------------------------------------------------------
\newcommand{\HRule}{\rule{\linewidth}{0.5mm}} % Command to make the lines in the title page

\setlength\parindent{0pt} % Removes all indentation from paragraphs

\begin{document}

%----------------------------------------------------------------------------------------
%	TITLE PAGE
%----------------------------------------------------------------------------------------
%\frontmatter % Use Roman numerals for page numbers

%\begin{center}

%

\title{
\begin{center}
\href{http://www.bradley.edu}{\includegraphics[height=0.5in]{figs/logoBU1-Print}}
\vskip10pt
\HRule \\[0.4cm]
{\Huge \bfseries Meeting Minutes \\[0.5cm] \Large BEMOSS and Its Enhanced Applications}\\[0.4cm] % Degree
\HRule \\[1.5cm]
\end{center}
}
\author{\Huge Brian Lauer \\ \\ \LARGE blauer@mail.bradley.edu \\[2cm]} % Your name and email address
\date{Beginning April 2019} % Beginning date
\maketitle

%\maketitle % Title page

\printindex
\tableofcontents % Table of contents
\newpage % Start lab look on a new page

\begin{addmargin}[0cm]{0cm} % Makes the text width much shorter for a compact look

\pagestyle{scrheadings} % Begin using headers

%----------------------------------------------------------------------------------------
%	LAB BOOK CONTENTS
%----------------------------------------------------------------------------------------

\labday{Friday, March 13, 2018}

\experiment{Notation}
%% Added by Dr. Miah 
Throughout this document, the vectors (matrices) will be denoted by lowercase (uppercase) bold letters while the lowercase non-bold letters will denote scalar quantities. Sets will be denoted by calligraphic letters. For positive integers $m,n>0,$ $\mathbb{R}^n(\mathbb{R}^{m\times n})$ denotes $n$--dimensional column vector ($m\times n$--dimensional matrix) with entries taken from a set of real numbers $\mathbb{R}.$ $(\cdot)^T$ denotes the transposition of quantity $(\cdot).$ The standard Euclidean norm of the vector  $\mathbf{x}\in\mathbb{R}^n$ and the matrix $\mathbf{A}$ are given by $\Vert x \Vert = \left(\sum_{i=1}^n\lvert x_i\rvert^2\right)^{1/2}$ and $\Vert \mathbf{A} \Vert = \left(\sum_{i=1}^m\sum_{j=1}^n\lvert a_{ij}\rvert^2\right)^{1/2}$  with $x_i,a_{ij}$ being the entries of $\mathbf{x}$ and $\mathbf{A},$ respectively. The scalar products of quantities $\mathbf{x},
\mathbf{y}\in\mathbb{R}^n$ and $\mathbf{A}, \mathbf{B}\in\mathbb{R}^{m\times n},$ are given by %
%
\begin{align*}
  \mathbf{x}^T\mathbf{y} = \sum_{i=1}^nx_iy_i~~\mathrm{and}~~\mathbf{A}\cdot \mathbf{B} =\mathrm{Tr}\left(\mathbf{A}^T\mathbf{B}\right) = \mathrm{Tr}\left(\mathbf{A}\mathbf{B}^T\right),
\end{align*}
%
respectively, where $\mathrm{Tr}(\cdot)$ is the trace of matrix $(\cdot).$ Clearly, $\mathrm{Tr}(\mathbf{A}^T\mathbf{A}) = \Vert \mathbf{A}\Vert^2.$ %
%
% ----------------------------------------------------------------------------------------

Example of citing a paper. ``Authors in~\cite{Martinelli2015-Robot} $\ldots$''


%----------------------------------------------------------------------------------------
%	LAB BOOK CONTENTS
%----------------------------------------------------------------------------------------
\labday{Friday, April 26, 2019}
%----------------------------------------------------------------------------------------
Dr. Miah will write a research grant proposal under BEMOSS and is due May 3.
\bigbreak\noindent
All documentation will be maintained in a github repository for this project.
I need to create a Github account and send my username to Dr. Miah. A Google Drive with the same name as the Github repository will be created name DocsBEMOSS.
\bigbreak\noindent
I need to talk to Mr. Mattus about getting a laptop for research. It must be installed with Ubuntu administrative privileges. 
\bigbreak\noindent
Send him an email:
\bigbreak\noindent
Hello Mr. Mattus, I will be working on a project with Dr. Miah. For that we would like to request a laptop available in the department if possible so that we can install Ubuntu operating system. 
\bigbreak\noindent
Try to go see him in the afternoon.
\bigbreak\noindent
I need to learn the following Github Bash terminal commands:
\begin{verbatim}
Git add
Git commit -m "message"
Git remote add origin 'url'
\end{verbatim}
To get a better undestanding of what is going on I need to go through Github tutorials.

%\item Upload latex file to Drive
\labday{Thursday, May 02, 2019}
I met with Reece Bachman, Jordan Ingram, and Robert O'Malley in the lab with the intention of filming a video of the BEMOSS installation from start to finish. However, I misinterpreted Dr. Miah's email and thought we were going to install on the current laptop they have in the lab. Instead, I was expected to have received one from Mr. Mattus with Ubuntu installed. Robert showed me the motor setup with the Zigbee module, L298N dual H bridge module, and buck-boost converter. Later, Jordan demonstrated his work with the HVAC controller.

\labday{Friday, May 03, 2019}
I still am in need of a laptop, so I must email Mr. Mattus once again on Monday, May 6.
\bigbreak\noindent
Dr. Miah showed me some of the Github commands to upload tex and pdf files to the Github repo. We added the meeting minutes files \texttt{meeting.tex} and \texttt{meeting.pdf}.

\labday{Friday, May 10, 2019}
Today, I met with Bob and Jordan to film the installation of BEMOSS on my borrowed laptop. After Bob attempted to install BEMOSS twice, he came to the conclusion that I installed the wrong version of Ubuntu on my machine. BEMOSS requires 16.04.5 not 16.04.6 that I had installed before the first installation. After another attempt at installing the software on Ubuntu 16.04.5, the same issue persisted involving three python modules not installing.
\end{addmargin}

%----------------------------------------------------------------------------------------
%	BIBLIOGRAPHY
%----------------------------------------------------------------------------------------


\bibliographystyle{plain}
\bibliography{bib/seniorProject2017.bib}


% \begin{thebibliography}{9}

% \bibitem{lamport94}
% Leslie Lamport,
% \emph{\LaTeX: A Document Preparation System}.
% Addison Wesley, Massachusetts,
% 2nd Edition,
% 1994.

% \end{thebibliography}

%----------------------------------------------------------------------------------------

\end{document}


%%% Local Variables:
%%% mode: latex
%%% TeX-master: t
%%% End:
